\documentclass[final, pdftex, 12pt]{../../template/epsilonj}

\usepackage{cwpuzzle}
\renewcommand\PuzzleFont{\rm\normalsize}
\renewcommand\PuzzleClueFont{\normalsize}
\renewcommand\PuzzleNumberFont{\scriptsize}
\makeatletter
\renewcommand\Puzzle@Clue@@normal[3]{\textbf{#1.}~#3 }
\makeatother

\hyphenation{си-с-те-му}

\begin{document}

\section*{Кроссворд №\,1}
\PuzzleUnitlength=16pt
\begin{Puzzle}{28}{20}%
|{}  |{}  |{}  |[1]q   |{}  |{}  |{}  |{}  |{}  |{}  |{}  |{}  |{}  |{}  |{}  |{}  |{}  |{}  |{}  |{}  |{}  |{}  |{}  |{}  |{}  |{}  |{}  |{}  |.
|{}  |[2]q   |{}  |q   |{}  |{}  |{}  |{}  |{}  |{}  |{}  |{}  |[3]q   |{}  |{}  |{}  |{}  |{}  |{}  |{}  |{}  |{}  |{}  |{}  |{}  |{}  |{}  |{}  |.
|{}  |[4]{в} |{е} |{л} |{и} |{к} |{о} |{б} |[5]{р} |{и} |{т} |{а} |{н} |{и} |{я} |{}  |{}  |{}  |{}  |[6]q   |{}  |{}  |{}  |{}  |{}  |{}  |{}  |{}  |.
|{}  |q   |{}  |q   |{}  |{}  |{}  |{}  |q   |{}  |{}  |{}  |q   |{}  |{}  |{}  |{}  |{}  |[7]q   |q   |q   |q   |q   |q   |q   |{}  |{}  |{}  |.
|{}  |q   |{}  |q   |{}  |{}  |{}  |{}  |q   |{}  |{}  |{}  |q   |{}  |{}  |{}  |{}  |{}  |{}  |q   |{}  |{}  |{}  |{}  |{}  |{}  |{}  |{}  |.
|{}  |q   |{}  |q   |{}  |{}  |{}  |{}  |q   |{}  |{}  |{}  |q   |{}  |{}  |{}  |{}  |{}  |[8]q   |q   |q   |q   |q   |q   |q   |q   |q   |q   |.
|{}  |q   |{}  |q   |{}  |{}  |{}  |{}  |q   |{}  |{}  |{}  |q   |{}  |{}  |{}  |{}  |{}  |{}  |q   |{}  |{}  |{}  |{}  |{}  |{}  |{}  |{}  |.
|{}  |q   |{}  |q   |{}  |{}  |{}  |{}  |q   |{}  |[9]q   |q   |q   |q   |q   |q   |q   |q   |q   |q   |q   |q   |q   |[10]q   |q   |q   |{}  |{}  |.
|{}  |{}  |{}  |{}  |{}  |{}  |{}  |{}  |q   |{}  |{}  |{}  |q   |{}  |{}  |{}  |{}  |{}  |{}  |q   |{}  |{}  |{}  |q   |{}  |{}  |{}  |{}  |.
|{}  |{}  |{}  |{}  |{}  |{}  |{}  |{}  |q   |{}  |{}  |{}  |{}  |{}  |{}  |{}  |[11]q   |{}  |{}  |q   |{}  |{}  |{}  |q   |{}  |{}  |{}  |{}  |.
|{}  |{}  |{}  |{}  |{}  |{}  |{}  |{}  |q   |{}  |[12]q   |{}  |{}  |{}  |{}  |{}  |q   |{}  |{}  |{}  |{}  |{}  |{}  |q   |{}  |{}  |{}  |{}  |.
|{}  |{}  |{}  |{}  |{}  |{}  |{}  |{}  |[13]q   |q   |q   |q   |q   |q   |q   |q   |q   |q   |{}  |[14]{а} |{в} |{с} |{т} |{р} |{а} |{л} |{и} |{я} |.
|{}  |{}  |[15]q   |{}  |{}  |[16]q   |{}  |{}  |q   |{}  |q   |{}  |{}  |{}  |{}  |{}  |q   |{}  |{}  |{}  |{}  |{}  |{}  |q   |{}  |{}  |{}  |{}  |.
|{}  |{}  |q   |{}  |{}  |q   |{}  |{}  |q   |{}  |q   |{}  |{}  |{}  |{}  |[17]q   |q   |q   |q   |q   |{}  |{}  |{}  |q   |{}  |{}  |{}  |{}  |.
|[18]q   |q   |q   |q   |q   |q   |q   |q   |q   |{}  |q   |{}  |{}  |{}  |{}  |{}  |{}  |{}  |{}  |{}  |{}  |[19]q   |q   |q   |q   |q   |q   |{}  |.
|{}  |{}  |q   |{}  |{}  |q   |{}  |{}  |{}  |{}  |q   |{}  |{}  |{}  |{}  |{}  |{}  |{}  |{}  |{}  |{}  |{}  |{}  |q   |{}  |{}  |{}  |{}  |.
|{}  |{}  |q   |{}  |{}  |q   |{}  |{}  |{}  |{}  |q   |{}  |{}  |{}  |{}  |{}  |{}  |{}  |{}  |{}  |{}  |{}  |{}  |q   |{}  |{}  |{}  |{}  |.
|{}  |{}  |q   |{}  |{}  |q   |{}  |{}  |{}  |[20]{г} |{а} |{н} |{а} |{}  |{}  |{}  |{}  |{}  |{}  |{}  |{}  |{}  |{}  |{}  |{}  |{}  |{}  |{}  |.
|{}  |{}  |q   |{}  |{}  |q   |{}  |{}  |{}  |{}  |q   |{}  |{}  |{}  |{}  |{}  |{}  |{}  |{}  |{}  |{}  |{}  |{}  |{}  |{}  |{}  |{}  |{}  |.
|{}  |{}  |q   |{}  |{}  |{}  |{}  |{}  |{}  |{}  |{}  |{}  |{}  |{}  |{}  |{}  |{}  |{}  |{}  |{}  |{}  |{}  |{}  |{}  |{}  |{}  |{}  |{}  |.
\end{Puzzle}
\begin{PuzzleClues}{\textbf{По горизонтали:}}
\Clue{4}{Великобритания}{Страна, разработавшая основной принцип англо"=американской модели.}
\Clue{7}{йй}{В честь какой страны названы известные шорты?}
\Clue{8}{йй}{То, к чему стремятся не прибегать в странах, где используется англо"=американская и континентальная модели.}
\Clue{9}{йй}{Какой подход характерен для англоязычных стран?}
\Clue{13}{йй}{Какой орган вырабатывает основные учетные принципы?}
\Clue{14}{Австралия}{Родина кенгуру.}
\Clue{17}{йй}{Страна, в которой больше всего специалистов, занимающихся бухгалтерским учётом.}
\Clue{18}{йй}{Страна, которой подходит формула «от теории "--- к практике».}
\Clue{19}{йй}{Страна, в честь которой назван головной убор.}
\Clue{20}{йй}{Япония, Гана, Швеция. Какая из этих стран используется англо"=американскую модель?}
\end{PuzzleClues}
\begin{PuzzleClues}{\textbf{По вертикали:}}
\Clue{1}{ETA}{Страна"=остров, использующая англо"=американскую систему.}
\Clue{2}{ETA}{Какая система бухгалтерской записи в целях исчисления финансового результата используется в~континетальной модели?}
\Clue{3}{ETA}{Внешний пользователь, на которого ориентирована англо"=американская система.}
\Clue{5}{ETA}{Принцип, характерный для англо"=американской модели.}
\Clue{6}{ETA}{Измеритель, используемый США и другими странами для синтетического учёта имущества.}
\Clue{10}{ETA}{Название одной из стран, где зародилась англо"=американская система.}
\Clue{11}{ETA}{Одна из стран англо"=американской модели.}
\Clue{12}{ETA}{Модель, позволяющая осуществлять контроль за сохранением ценностей.}
\Clue{15}{ETA}{Страна с~англо"=американской моделью учёта.}
\Clue{16}{ETA}{Эта страна недавно вышла из сферы влияния Великобритании.}
\end{PuzzleClues}
\par\bigskip
Подошло:

1.~Ирландия.
2.~Двойная.
3.~Инвестор.
4.~Великобритания.
5.~Рекапитуляция.
6.~Ценность.
7.~Бермуды.
10.~Нидерланды.
11.~Бенин.
12.~Исламская \textit{(использует рыночные цены для оценки)}.
14.~Австралия.
15.~Гондурас, Сингапур или Танзания.
16.~Гонконг.
17.~Индия.
18.~Финяндия.
19.~Панама.
20.~Гана.
\par\medskip
Не подошло:

8.~Деинфляция \textit{(шутка; конечно же, «дефлирование»; это свойственно южноамериканской модели)}.
9.~Агосударствление \textit{(приставка «а-» означает отрицание; это шутка, но, тем не менее, для англоязычных стран характерно отдаление от государственного контроля и рыночная ориентированность!)}.
13.~Совет по МСФО \textit{(вообще не подходит)}. 

\end{document}
