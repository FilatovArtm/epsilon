Весёлый уголок

Как отцы науки коэффициенты подбирали
«
Мои данныя состояли изъ 930~наблюденій роста совершеннолѣтнихъ дѣтей и ихъ прямыхъ восходящихъ родственниковъ, числомъ 205~составлявшихъ. Всякій разъ я превращалъ высоту женскаго стана въ эквивалентъ мужескаго и использовалъ ихъ въ преобразованномъ видѣ, дабы не вызвать упрековъ, происходящихъ отъ наличествованія разницы въ ростахъ половой природы, буде я говорилъ"=бы о среднихъ. Множитель, использованный мною, составлялъ 1,08, что равносильно прибавленію чуть менѣе чѣмъ одной двѣнадцатой части къ росту каждой женщины. Сей множитель ненамного отличается отъ оныхъ, использованныхъ иными антропологами, кои къ тому-же сами малости въ различіяхъ множителей имѣютъ; какъ-бы то ни было, онъ подходитъ къ моимъ даннымъ лучше, нежели 1,07 или 1,09. Итоговый результатъ никоимъ образомъ не относится къ тѣмъ, что зависятъ отъ этихъ минутныхъ деталей, ибо такъ сталось, что изъ-за ошибочнаго указанія расчетчикъ, которому я попервости ввѣрилъ цифры, использовалъ немного другой множитель, однако результатъ вышелъ практически одинъ въ одинъ. 
»
Фрэнсисъ Гальтонъ, «Регрессированіе къ посредственному при наслѣдованіи ростовъ», 1886.

