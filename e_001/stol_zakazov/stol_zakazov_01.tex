\documentclass[final,pdftex]{../../template/epsilonj}

\RequirePackage{graphicx}
\RequirePackage[colorlinks,citecolor=blue,urlcolor=blue]{hyperref}

\addbibresource{../../template/epsilon.bib}

\begin{document}

\setcounter{page}{59}

% \microtypesetup{protrusion=false, expansion=false}
\begin{frontmatter}
\title{Стол заказов}
\runtitle{Стол заказов}

\begin{aug}
\author{\imya{Борис} \fam{Демешев}}%


%\runauthor{Борис Демешев}

%\address{НИУ ВШЭ, Москва.}
\end{aug}

%\begin{abstract}
%\end{abstract}

%\begin{keyword}
%	\kwd{статистика}
%	\kwd{вопросы}
%	\kwd{ответы}
%	\kwd{интернет}
%\end{keyword}

\end{frontmatter}

% \microtypesetup{protrusion=true, expansion=true}

% \section{xxx}



Для того, чтобы выходили новые номера <<Эпсилон>>, нужны новые статьи. И побольше, побольше\ldots Присоединяйтесь к нам, пишите как мы, пишите лучше нас!




\section{Смысловая часть}

Статья должна быть интересной и не содержать отклонений от здравомыслия! Она может относиться к~эконометрике, теории игр, динамической оптимизации, анализу данных, теории вероятностей или математической статистике, а также всему, что понадобится впредь. Хорошо бы, чтобы в статье были красивые картинки! Если у вас есть свои идеи --- предлагайте, обсудим. Если вы хотите, чтобы мы написали про что-то статью, спросите! Если есть желание написать статью для <<Эпсилон>>, а идей нет, то мы предлагаем готовые сюжеты для методических статей-проектов и готовы оказать поддержку в написании:

\begin{itemize}
\item Векторно-матричное дифференциирование с примерами. 

Как взять производную по вектору? По матрице? Среди примеров могут быть: оценка  ковариационной матрицы c~помощью ML, МНК, LDA, PCA, канонические корреляции.
\item Линейный дискриминантный анализ по чесноку. 

Что это за зверь? Какова его связь с~логит-регрессией? Примеры задач. Честный вывод формул. 

\item Как мы порвали всех во втором туре универсиады по метрике. 

Условие задачи и решение с кодом \proglang{R}.

\item Частная корреляция. 

Отличия от обычной. Два подхода к расчёту: через проецирование и уравнение регрессии. Эквивалентность через теорему Фриша-Вау.

\item Теорема Фриша-Вау с~примерами.

Геометрическое доказательство Фриша-Вау. Примеры: регрессия на константу, частная корреляция

\item Как строить карты России в~\proglang{R}?

Больше примеров карт, хороших и разных!

\item Пакет для обработки данных RLMS в \proglang{R}.

Написать функцию для автоматического объединения данных по индивидам из разных волн на основании родственных связей. И описать с~примерами

\item Пакет со списком источников экономических данных по России и вспомогательными функциями для работы с~ними.

Смутная идея, но вдруг кто вдохновится?


\item Ещё раз подчеркнём: это идеи сюжетов для тех, кто хочет написать статью, но не знает за что взяться. Мы только <<за>> другие смелые идеи!

\end{itemize}



В методических статьях главное --- чтобы и ежу было понятно! Разумно включать простые примеры и упражнения для читателя. Если дело касается работы с реальными данными, то очень желателен код \proglang{R}.

\section{Техническая часть}


Мы принимаем статьи, написанные с помощью \LaTeX, языка разметки маркдаун и грамотного программирования (literate programming) с использованием \proglang{R} (форматы \code|Rmd|, \code|Rnw|). Если у вас есть особо ценная статья написанная в другом формате, пишите нам, обсудим. С~исходными текстами статей первого номера можно ознакомиться, глянув репозиторий \url{https://github.com/bdemeshev/epsilon/tree/master/e_001}.

Пожалуйста, используйте \LaTeX{} правильно. Сейчас мы ведём переговоры с Биллом Гейтсом, Карлосом Слимом и Уорреном Баффетом о~спонсировании работы штата корректоров, а пока переговоры продолжаются мы просим будущих авторов:

\begin{enumerate}
\item Для выносных формул не используйте окружение \code|$$ $$|. Используйте \code|\[ \]|.
\item Не забывайте неразрывный пробел, \code|~|, после короткого предлога. Например, \code|в~Москве|.
\item Не забывайте длинное тире. Не забывайте, что оно пишется с~помощью трёх коротких чёрточек, \code|---|.
\item Языки программирования или статистический софт записывайте с~помощью \code|\proglang{}|. Например, \code|\proglang{R}|.
\item Пакеты для \proglang{R} или другой среды записывайте с~помощью \code|\pkg{}|. Например, \code|\pkg{dplyr}|.
\item Для отдельно стоящих команд \proglang{R} или другого пакета используйте \code/\code||/. Например, \code/\code|model <- lm(data=cars, dist~speed)|/.
\item Любите и уважайте букву <<ё>>.
\end{enumerate}



\end{document}
