\documentclass[final,pdftex]{../../template/epsilonj}

\RequirePackage{graphicx}
\RequirePackage[colorlinks,citecolor=blue,urlcolor=blue]{hyperref}

\addbibresource{../../template/epsilon.bib}

\begin{document}

% \microtypesetup{protrusion=false, expansion=false}
\begin{frontmatter}
\title{Эконометрика: типичные ошибки студентов и~аспирантов}
\runtitle{Эконометрика: типичные ошибки студентов и~аспирантов}

\begin{aug}
\author{\imya{Борис} \fam{Демешев}}%
\author{\imya{Кирилл} \fam{Фурманов}}

\runauthor{Демешев Б. Б., Фурманов К. К.}

\address{НИУ ВШЭ, Москва.}
\end{aug}

\begin{abstract}
В~этой короткой статье перечислены ошибки, наиболее часто допускаемые студентами и~аспирантами при интерпретации эконометрических моделей, написании работ и~презентации своих результатов окружающим.
\end{abstract}

\begin{keyword}
	\kwd{эконометрическая культура}
	\kwd{статистическая мудрость}
	\kwd{довольная комиссия}
	\kwd{распространённые ошибки}
\end{keyword}

\end{frontmatter}

% \microtypesetup{protrusion=true, expansion=true}

\section{Советы Кирилла Фурманова}

\begin{enumerate}
	\item Коэффициенты в~регрессии показывают наличие лишь статистической взаимосвязи. Причинно"=следственная интерпретация часто ошибочна. Она возможна в~некоторых случаях "--- например, когда данные получены в~результате эксперимента.
	\item Если нулевая гипотеза не отвергается, это не означает, что она верна. Корректно говорить, что недостаточно данных, чтобы отвергнуть $\hypo_0$, или что данные не противоречат $\hypo_0$. При этом они ещё много чему могут не противоречить.
	\item Нужно понимать соответствие между содержательной гипотезой, формулируемой без статистических терминов, и~формулировкой в~терминах нулевой"=альтернативной гипотезы. 
	\item «Мы говорим "--- Ленин, подразумеваем "--- партия, мы говорим "--- партия, подразумеваем "--- Ленин». (\cite{mayakovskiy}.) Мы говорим, что проверяем гипотезу о~значимости регрессии, хотя на самом деле проверяем гипотезу о~незначимости регрессии, то есть $\hypo_0\colon \beta_2=\beta_3=\ldots=\beta_k=0$. 
	\item Если правильно интерпретировать коэффициенты, то разумными и~верными могут оказаться одновременно несколько разных моделей. При этом все оценки коэффициентов будут несмещёнными! Например, и~регрессия игрека на икс, и~регрессия игрека на икс и~зет могут быть осмысленны и~интересны. Коэффициент при иксе в~первой модели показывает, на сколько в~среднем изменяется игрек, когда икс меняется на единицу, а~во~второй "--- насколько в~среднем изменяется игрек, когда икс меняется на единицу, а~зет не изменяется.
	\item Значимость "--- это не то же самое, что существенность. Коэффициент может быть значимым, но совершенно бесполезным. Если месячная зарплата мужчин и~женщин значимо отличается, но это отличие составляет два рубля, то можно считать, что его нет. Возможно домножать коэффициент на стандартную ошибку регрессора или на квантили.
\end{enumerate}

\section{Советы Бориса Демешева}

\begin{enumerate}
	\item Больше графиков! Работа по эконометрике без картинок скучна и~бессмысленна. Если сомневаешься, нужно ли построить ещё один график, значит, нужно. Графиков много разных, не бойся экспериментировать! Например, на одном графике можно осмысленно изобразить 60 временных рядов (см.~\cite{mvtsplot}). 
	\item Рассказывай презентацию для идиотов. Методов и~моделей слишком много. Во~время презентации исходи из предположения, что комиссия не знает эту тему. Выбирай простые примеры, а~не рассказывай про общий случай. Меньше буковок на слайдах! 
	\item «Ларису Ивановну хочу!» "--- «Хочу построить модель» "--- это бяка! Хочу заработать миллион баксов и~поэтому хочу предсказывать, сколько будет стоить завтра доллар: это может быть кому-то интересно. Хочу проверить содержательную гипотезу такую-то "--- тоже прикольно.
	\item Зачастую не нужны сложные модели. Всегда проверяй сложную модель против самой простейшей, которая приходит в~голову. Например, качество прогнозов во~временных рядах стоит проверить против модели «завтра будет так же, как сегодня». 
	\item После написания курсовой, ВКР, диссертации напиши для себя мораль. Там, конечно, аршинными буквами будет «\textsc{в~следующем году я~начну писать до Нового года}». Чему тебя научила ВКР? Ещё можно написать протокол воспроизведения всех регрессий и~обработки данных. Полезно выложить в~публичный доступ. 
	\item Осваивай открытый софт: R\textbf{\textbf{}}, gretl, \LaTeX, Markdown, Python, SQL и~ещё куча страшных слов! Это бесплатно, и~сообщество просто огромное. Модно, стильно, молодёжно! 
	\item А~ты знаешь, что такое $p$-value? Учи матстат! 
	\item Мелочи по представлению результатов, способные вызвать праведный гнев членов комиссии:
	\begin{enumerate}
		\item \textit{Техническая подготовка.} Проверь флешку, выложи файл в~интернет заранее, чтобы, если флешка не сработает, его можно было быстро скачать. Будь готов к~мелочам вроде проектора, не отличающего красного от розового, яркого солнца, и~полям слайда, вылезающим за доску. 
		\item \textit{Отсутствие номера на слайде.} А~покажи-ка мне слайд, ну, этот\ldotst{} на котором\ldotst{} Делай номер внизу слайда.
		\item \textit{Неподписанные оси на графике.} Принцип идеального графика: идеальный график можно понять, не читая оставшуюся часть работы. Подписывай оси, приводи единицы измерения, расшифровывай названия переменных. Чем больше нужно устных комментариев к~графику, чтобы понять его, тем хуже график.
		\item \textit{Семь знаков после запятой.} По некоторым данным (\cite{piraha2008}), люди народности пираха (Бразилия) считают так: один, два и~много. Пираха знают толк в~знаках после запятой!
	\end{enumerate}
\end{enumerate}

\printbibliography

\end{document}
