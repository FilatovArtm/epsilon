\documentclass[final,pdftex]{../../template/epsilonj}

\RequirePackage{graphicx}
\RequirePackage[colorlinks,citecolor=blue,urlcolor=blue]{hyperref}

\addbibresource{../../template/epsilon.bib}

\begin{document}

\setcounter{page}{2}

% \microtypesetup{protrusion=false, expansion=false}
\begin{frontmatter}
\title{Вступительное слово}
\runtitle{Вступительное слово}

\begin{aug}
\author{\imya{Николай} \fam{Пильник}}%


%\runauthor{Николай Пильник}

%\address{НИУ ВШЭ, Москва.}
\end{aug}

%\begin{abstract}
%\end{abstract}

%\begin{keyword}
%	\kwd{статистика}
%	\kwd{вопросы}
%	\kwd{ответы}
%	\kwd{интернет}
%\end{keyword}

\end{frontmatter}

% \microtypesetup{protrusion=true, expansion=true}

% \section{xxx}


Дорогие читатели! Мы рады представить вам первый выпуск околонаучно-методического эконометрико-сатирического журнала <<Эпсилон>>, 
который, как мы надеемся, превратится в регулярное и в меру серьезное издание. 
Для нас этот журнал --- попытка
собрать в одном месте идеи, наблюдения и наработки в интересующих нас
математических областях экономической науки и ее соседей.

Мы будем писать о темах, которые не только крайне интересны нам, но могут представлять интерес, как для преподавателей, так и для студентов. 
В первую очередь, мы ориентируемся на сотрудников и студентов Высшей Школы Экономики, хотя такая постановка на данный
момент во многом связана не только с нашей <<ведомственной
принадлежностью>>, но и с тем, что интересы этой аудитории мы представляем лучше всего.

Журнал <<Эпсилон>> задумывается нами как регулярное издание, не
исключающее специальных дополнительных выпусков. 
% Мы надеемся, что выбранная нами манера подачи материала и составленные с особой тщательностью задачи привлекут внимание студентов, небезразличных к собственным научным достижениям и открытиям. 

% Научным сотрудникам и
% преподавателям, как нам кажется, будут любопытны статьи с методическими
% рекомендациями, которые, по мнению редакции, заслуживают внимания. 

Особого отношения мы ожидаем в отношении постоянных
спутников читателя на страницах нашего журнала --- бобров.


\end{document}
