\documentclass[11pt]{article}

\usepackage{epsilonj}
\RequirePackage{graphicx}
\RequirePackage[colorlinks,citecolor=blue,urlcolor=blue]{hyperref}

\addbibresource{../../template/epsilon.bib}

\theoremstyle{definition} % definition/plain/remark
\newtheorem{zadacha}{Задача}


\begin{document}

\TITLE{Задачи}
\SHORTTITLE{Задачи}

\AUTHOR{Фольклор и коллектив кафедры}{}
\SHORTAUTHOR{Фольклор и коллектив кафедры}

\DoFirstPageTechnicalStuff


\begin{abstract}
Прикольные задачи по теории вероятностей, теории игр, динамической оптимизации, а заодно и здравому смыслу!
\end{abstract}

\begin{keyword}
взрыв мозга, только хардкор, слабо
\end{keyword}


\begin{zadacha}
Злобный Дракон поймал принцесс Настю и Сашу и посадил в разные башни. Перед каждой из принцесс Злобный Дракон подбрасывает один раз правильную монетку. А дальше даёт каждой из них шанс угадать, как выпала монетка у её подруги. Если хотя бы одна из принцесс угадает, то Злобный Дракон отпустит принцесс на волю. Если обе принцессы ошибутся, то они навсегда останутся у него в заточении.

Подобная практика у Злобного Дракона исследователями была отмечена уже давно, поэтому принцессы имели достаточно времени договориться на случай вероятного похищения.

Как следует поступать принцессам при подобных похищениях?
\end{zadacha}

\begin{zadacha}
Удав-Пустынник любит программировать на \proglang{python} и есть французские багеты\footnote{«Удав из которого говорит кролик, "--- это не тот удав, который нам нужен». \citep{iskander:kroliki}}. Длина французского багета равна 1 метру. За один заглот Удав-Пустынник заглатывает кусок случайной длины равномерно распределенной на отрезке $[0;1]$. Для того, чтобы съесть весь багет удаву потребуется случайное количество $N$ заглотов. 
\begin{enumerate}
\item Найдите $\E(N)$ и $\Var(N)$
\item Как поменяются ответы, если багет имеет длину 2~метра?
\end{enumerate}
\end{zadacha}

\begin{zadacha}
Ефросинья подкидывают правильную монетку неограниченное количество раз. 

\begin{enumerate}
\item Сколько в среднем нужно сделать бросков до появления последовательности ОРОР? 
\item А до появления последовательности РОРР?
\item Какова вероятность того, что ОРОР появится раньше РОРР?
\end{enumerate}

\end{zadacha}

\begin{zadacha}
Эконометресса Барбара оценивает с помощью МНК модель $y_t=\beta x_t+\varepsilon_t$. Ошибки $\varepsilon_t$ независимы, имеют нулевое среднее и постоянную дисперсию, регрессоры известны и равны $x_t=1/2^t$. 
\begin{enumerate}
\item Получит ли Барбара состоятельную оценку для $\beta$?
\item Эконометресса Виолетта оценивает с помощью взвешенного МНК ту же модель, однако ошибочно предполагает, что имеет место гетероскедастичность вида $\Var(\varepsilon_t)=\sigma^2 x_t^2$. Получит ли Виолетта состоятельную оценку для $\beta$?
\end{enumerate}
\end{zadacha}

\begin{zadacha}
Трое заядлых игроков в покер сидят в чате. Предложите процедуру раздачи карт, при которой каждый игрок знает свои карты и не знает карт соперника. Игроки абсолютно рациональны и обладают безграничными вычислительными возможностями, поэтому использование кодов с открытым ключом (типа RSA) недопустимо. В чате можно посылать сообщения, адресованные как всем сразу, так и конкретному лицу.
\end{zadacha}


\printbibliography


\end{document}
