\documentclass[final,pdftex]{../../template/epsilonj}

\RequirePackage{graphicx}
\RequirePackage[colorlinks,citecolor=blue,urlcolor=blue]{hyperref}

\newcommand{\specialcell}[2][c]{\begin{tabular}[#1]{@{}c@{}}#2\end{tabular}}

\begin{document}

% \microtypesetup{protrusion=false, expansion=false}
\begin{frontmatter}
\title{Задачи}
\runtitle{Задачи}

\begin{aug}
\author{\imya{Борис} \fam{Демешев}}%

\runauthor{Борис Демешев}

\address{НИУ ВШЭ, Москва.}
\end{aug}

\begin{abstract}
Прикольные задачи по теории вероятностей, теории игр, динамической оптимизации, а заодно и здравому смыслу!
\end{abstract}

\begin{keyword}
	\kwd{взрыв мозга}
	\kwd{только хардкор}
	\kwd{слабо}
\end{keyword}

\end{frontmatter}

% \microtypesetup{protrusion=true, expansion=true}

%\section{Поехали}

\begin{enumerate}

\item Злобный Дракон поймал принцесс Настю и Сашу и посадил в разные башни. Перед каждой из принцесс Злобный Дракон подбрасывает один раз правильную монетку. А дальше даёт каждой из них шанс угадать, как выпала монетка у её подруги. Если хотя бы одна из принцесс угадает, то Злобный Дракон отпустит принцесс на волю. Если обе принцессы ошибутся, то они навсегда останутся у него в заточении.

Подобная практика у Злобного Дракона исследователями была отмечена уже давно, поэтому принцессы имели достаточно времени договориться на случай вероятного похищения.

Как следует поступать принцессам при подобных похищениях?
\item Удав Анатолий любит французские багеты. Длина французского багета равна 1 метру. За один заглот Удав Анатолий заглатывает кусок случайной длины равномерно распределенной на отрезке $[0;1]$. Для того, чтобы съесть весь багет удаву потребуется случайное количество $N$ заглотов. 
\begin{enumerate}
\item Найдите $\E(N)$ и $\Var(N)$
\item Как поменяются ответы, если багет имеет длину 2 метра?
\end{enumerate}

\item Ефросинья подкидывают правильную монетку неограниченное количество раз. 

\begin{enumerate}
\item Сколько в среднем нужно сделать бросков до появления последовательности ОРОР? 
\item А до появления последовательности РОРР?
\item Какова вероятность того, что ОРОР появится раньше РОРР?
\end{enumerate}

\item Эконометресса Барбара оценивает с помощью МНК модель $y_t=\beta x_t+\varepsilon_t$. Ошибки $\varepsilon_t$ независимы, имеют нулевое среднее и постоянную дисперсию, регрессоры известны и равны $x_t=1/2^t$. 
\begin{enumerate}
\item Получит ли Барбара состоятельную оценку для $\beta$?
\item Эконометресса Виолетта оценивает с помощью взвешенного МНК ту же модель, однако ошибочно предполагает, что имеет место гетероскедастичность вида $\Var(\varepsilon_t)=\sigma^2 x_t^2$. Получит ли Виолетта состоятельную оценку для $\beta$?
\end{enumerate}

\item Трое заядлых игроков в покер сидят в чате. Предложите процедуру раздачи карт, при которой каждый игрок знает свои карты и не знает карт соперника. Игроки абсолютно рациональны и обладают безграничными вычислительными возможностями, поэтому использование кодов с открытым ключом (типа RSA) недопустимо. В чате можно посылать сообщения, адресованные как всем сразу, так и конкретному лицу.

\end{enumerate}




\end{document}
