\documentclass[10pt]{article}

\usepackage{epsilonj}

\RequirePackage{graphicx}
\RequirePackage[colorlinks,citecolor=blue,urlcolor=blue]{hyperref}

\newcommand{\specialcell}[2][c]{\begin{tabular}[#1]{@{}c@{}}#2\end{tabular}}

\begin{document}

\TITLE{Корреляция: простая, частная и условная}
\SHORTTITLE{Корреляция: простая, частная и условная}

\AUTHOR{Борис Демешев}{НИУ ВШЭ, Москва.}
\SHORTAUTHOR{Борис Демешев}

\DoFirstPageTechnicalStuff


\newtheorem{theorem}{Теорема}
\newtheorem{definition}{Определение}

\begin{abstract}
Корреляция "--- это способ описать силу линейной зависимости между двумя случайными величинами одним числом. Каков геометрический смысл корреляции? Что такое частная корреляция? Как связаны частная и условная корреляция? 
\end{abstract}

\begin{keyword}
корреляция, частная корреляция, условная корреляция, косинус, проекция
\end{keyword}



\section{Корреляция по-русски}

Обычно в учебниках даётся такое определение корреляции

\[
\Corr(X,Y)=\frac{\Cov(X,Y)}{\sqrt{\Var(X)\Var(Y)}}.
\]

Естественно, возникает вопрос: «С какого перепугу? Почему это мы делим ковариацию на что-то там?»

Мы дадим определение корреляции словами:

\begin{definition}
Корреляция между случайными величинами $X$ и $Y$ показывает на сколько своих стандартных отклонений в среднем растёт случайная величина $Y$ при росте случайной величины $X$ на одно своё стандартное отклонение.
\end{definition}

А теперь из этого словесного определения мы получим формулу (...). Разложим величину $Y$ на два слагаемых. Первое слагаемое вбирает в себя всю ту часть $Y$, которая линейно зависит от $X$, а второе --- всё оставшееся:

\[
\frac{Y}{\sigma_Y}=\rho \cdot \frac{X}{\sigma_X} + \varepsilon
\]

В этой формуле видно, что с ростом $X$ на одно стандартное отклонений $\sigma_X$ правая часть изменится в среднем на $\rho$, и, следовательно, величина $Y$ в среднем изменится на $\rho \cdot \sigma_Y$. 

%\[
%Y=\beta \cdot X + \varepsilon
%\]
%
%Здесь $\beta$ показывает на сколько единиц в среднем растёт $Y$ при росте $X$ на одну единицу. 

Мы хотим, чтобы ... $\Cov(X,\varepsilon)=0$. 


\[
\Cov\left(X, \frac{Y}{\sigma_Y} - \rho \cdot \frac{X}{\sigma_X} \right) = 0
\]

По свойствам ковариации получаем

\[
\Cov(X,Y)/\sigma_Y=\rho \Cov(X,X)/\sigma_X
\]

И, тадам, выражаем корреляцию, $\rho$:

\[
\rho = \frac{\Cov(X,Y)}{\sigma_X \sigma_Y}
\]

\section{Геометрический смысл корреляции}

Длина случайной величины --- 

Картинка с корреляцией

\section{Корреляция и независимость}

\begin{theorem}
Случайные величины $X$ и $Y$ независимы тогда и только тогда, когда некоррелированы любые функции $f(X)$ и $g(Y)$.
\end{theorem}

Здесь про $\E(Y|X)$ ???

\section{Частная корреляция}

Определение.


Обозначение???

Геометрическая интерпретация

\section{Условная корреляция}



\end{document}


